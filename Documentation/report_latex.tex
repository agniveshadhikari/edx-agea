%Copyright 2014 Jean-Philippe Eisenbarth
%This program is free software: you can 
%redistribute it and/or modify it under the terms of the GNU General Public 
%License as published by the Free Software Foundation, either version 3 of the 
%License, or (at your option) any later version.
%This program is distributed in the hope that it will be useful,but WITHOUT ANY 
%WARRANTY; without even the implied warranty of MERCHANTABILITY or FITNESS FOR A 
%PARTICULAR PURPOSE. See the GNU General Public License for more details.
%You should have received a copy of the GNU General Public License along with 
%this program.  If not, see <http://www.gnu.org/licenses/>.

%Based on the code of Yiannis Lazarides
%http://tex.stackexchange.com/questions/42602/software-requirements-specification-with-latex
%http://tex.stackexchange.com/users/963/yiannis-lazarides
%Also based on the template of Karl E. Wiegers
%http://www.se.rit.edu/~emad/teaching/slides/srs_template_sep14.pdf
%http://karlwiegers.com
\documentclass{scrreprt}
\usepackage{listings}
\usepackage{underscore}
\usepackage[bookmarks=true]{hyperref}
\usepackage[utf8]{inputenc}
\usepackage[english]{babel}
\usepackage{graphicx}
\hypersetup{
    bookmarks=false,    % show bookmarks bar?
    pdftitle={Software Requirement Specification},    % title
    pdfauthor={Jean-Philippe Eisenbarth},                     % author
    pdfsubject={TeX and LaTeX},                        % subject of the document
    pdfkeywords={TeX, LaTeX, graphics, images}, % list of keywords
    colorlinks=true,       % false: boxed links; true: colored links
    linkcolor=blue,       % color of internal links
    citecolor=black,       % color of links to bibliography
    filecolor=black,        % color of file links
    urlcolor=purple,        % color of external links
    linktoc=page            % only page is linked
}%
\def\myversion{1.0 }
\date{}
%\title
\usepackage{hyperref}
\begin{document}

\begin{flushright}
    \rule{14cm}{4pt}\vskip1cm
    
        \Huge{EXCEL SHEET ASSESSMENT XBLOCK}\\
        
        \vspace{1.0cm}
        Enhancements to IIT BombayX Platform\\
        \vspace{1.3cm}
        Prepared by \\Shreya Saha\\Agnivesh Adhikari\\Shree Shubh\\
        \vspace{1.3cm}
       
        
        \today\\
    \vspace{0.7cm}
    Under the Guidance of\\
{\Huge Prof. D. B.Phatak}\\
{\normalsize Department of Computer Science and Engineering\\
Indian Institute of Technology, Bombay\\ }
\end{flushright}

\tableofcontents

\chapter{Acknowledgement}

It is our privilege to express our sincerest regards and deep
gratitude to all those who have helped us complete our project
successfully. It is the result of collective efforts of coordinators,
mentors and all the team members.\\
We deeply express our sincere thanks to Prof. D.B. Phatak for
selecting us and allowing us to work on the project “Enhancements to IIT BombayX Platform“.\\
We thank Mr. Awinash Awate for his constant impetus and
motivation that has helped us evolve better.\\
We would like to express deep gratitude to our project mentor,
Mr. Aparna Pansare, for her
valuable input, guidance, encouragement, whole-hearted
cooperation and constructive criticism throughout the duration
of our project. Her consistent efforts and guidance have
helped us develop a deep insight and understanding of various
concepts.\\
We would also like to thank Mr. Rahul Kharat and Mr. Bikas
Chhatri for making our stay here comfortable and for all their
administrative help.\\
We would also like to thank Miss. Priyanka Garg and Mr. Anand
Roy Choudhary for managing the uploading and downloading of files.\\
We would also like to mention Prashant ---- and Pravin --- for their support and help 
during the period of our internship.\\
Lastly, we whole heartedly thank all our other colleagues and
fellow interns for helping us with their critical advice and
support.

\chapter{Introduction}

\section{Current state of online learning:}
Massive Open Online Courses (MOOCs) is overpowered the area of online education everywhere.Professors from world class 
institutes are making online content *(and assessments) available both in free and (increasingly) paid courses. Renowned
institutions are accepting these credits as part of their curriculum.MOOC learning currently revolves around videos, text
and images used by students to grab the subjective knowledge which is then assessed by subjective,objective or numerical 
based tests.In the US, school education is undergoing a transformation. More and more schools are adopting hybrid models 
of learning over the traditional classroom based (rote) approach.Assessment of learning is a weak area especially in MOOCs 
where physical assessment is impossible. In the hybrid models as well, physical assessments (as against the e-Assessments) 
bring in a subjective element. An e-Assessment has the risk of large scale 'copying' thus is not an appropriate parameter of
judging the concepts of student.

\section{edX Platform}

edX is a massive open online course (MOOC) provider.It hosts online university level courses in a wide range of disciplines
to a worldwide student body, including some courses at no charge. It also conducts research into learning based on how people 
use its platform.EdX differs from other MOOC providers such as coursera or Udacity,in that it is a non profit organization and 
runs on the open edX open-source software.\\
The Massachusetts Institute of Technology and Harvard University created edX in May 2012.More than 70 schools,non profit 
organizations, and corporations offer to plan to offer courses on the edX website.As of 29 December,2016, edX has around 10
million students taking more than 1270 courses online.

\section{XBlocks}

XBlock is the SDK for the edX MOOC platform,written in python2 and announced and released publicly on March 14,2013.It aims 
to enable the global software developement community to participate in the construction of edx educational platform and the
next generation of online and blended courses.\\
XBlock is a component architecture that enables developers to create independent course components, or XBlocks, that are able 
to work seamlessly with other components in the construction and presentation of an online course. Course authors are able to 
combine XBlocks from a variety of sources — from text and video to sophisticated wiki-based collaborative learning environments
and online laboratories — to create rich engaging online courses.

\section{Objective}
The purpose of our project is to create an exam system in which the teacher uploads a question
paper written in the excel file in a certain format and the student uploads his answer file in the same format.
The answer file of the student is then automatically graded.The student can also see his progress in the progress bar.

\section{Purpose}
\begin{enumerate}
\item To enhance the platform of IIT BombayX.
\item To reduce the workload of the teacher when large number of students are present.
\end{enumerate}
\section{The future of learning in the world:}
In the future with advanced technology, we can allow not only excel files but also text,image and other possible files.

\section{Scope of Project}
This system will be Excel Sheet Management System for a School/University. This
system will be designed to maximize the examination productivity by providing tools to
assist in automating the correction of answer sheets and publishing grade process, which
would otherwise have to be performed manually.
This system can be quite useful in accounting exams and also in other exams which can involve excel comparisons.

\section{Glossary}
    

    
	    Open edx::Open edx is the open source platform supporting the edx courses. \\        
	    Student::Person who will give the exam and upload his answer sheet. \\        
	    Database::Collection of all the information monitored by this system. \\        
            Teacher::Person who is conducting the exam. \\        
            Software Requirements Specification::describes the functions of a system and its constraints. \\        
            User::Teacher or Student  \\       
            OpenPyxl::It is a Python library for reading and writing Excel 2010 xlsx files.\\
            XBlock::To create rich, engaging online courses, course authors must be able to combine components 
            from a variety of sources. XBlock, which is edX’s component architecture, makes this possible. 
            Courses are built hierarchically of pieces called XBlocks.



\section{References}
IEEE. IEEE Std 830-1998 IEEE Recommended Practice for Software Requirements
Specifications. IEEE Computer Society, 1998.

\chapter{Overview of the entire project}

\section{Use case Diagram}
\includegraphics[totalheight=10cm]{bruh}
\DeclareGraphicsExtensions{.pdf,.jpg,.png}


The system has two views-Teacher view and Student view.The teacher uploads an
annotated question template which is then downloaded by the student.The student fills in
the answers and uploads his answer template which is then automatically graded and can
be viewed both by the teacher and the student.

\section{Functional Requirements Specification-Student's View}

\begin{enumerate}
\item Teacher uploads annotated question file.
\item Student downloads the file (from the student view).
\item Student solves the problem, makes changes to the annotated file, and uploads the
answer script (from the student view).
\item Auto grader will run on the answer script. Grades will be computed.
\item The student can now view his grades.
\end{enumerate}

\section{Functional Requirements Specification-Teacher’s View}

\begin{enumerate}
\item Teacher upload the annotated question template and answer sheet.
\item Student download the annotated question paper.
\item Student solves the problem, makes changes to the annotated file, and uploads the
answer script (from the student view).
\item Auto grader will run on the answer script. Grades will be computed.
\item Both student and teacher can view the grade assigned to the student.
\item Teacher has the option to download the answer script.
\end{enumerate}

\section{Non-functional Requirements}

To use our system,the user must have edx platform installed in their local server.Also, the project 
preferebly runs on web browsers like google chrome and Mozilla Firefox. 

\chapter{Workflow of the entire system}

\section{Logical Structure of data}
\includegraphics[totalheight=10cm]{bruh2}
\DeclareGraphicsExtensions{.pdf,.jpg,.png}

\section{Steps to be followed by the student}

\begin{enumerate}
\item Log into the student's view and enter your course.\\
\includegraphics[totalheight=7cm]{ash}
\DeclareGraphicsExtensions{.pdf,.jpg,.png}

\item Choose your answer file through the browse button.\\
\includegraphics[totalheight=7cm]{mes}
\DeclareGraphicsExtensions{.pdf,.jpg,.png}\\
Now,upload it.

\newpage 

\item Your grade will be displayed along with the number of attempts you have left.\\
\includegraphics[totalheight=8cm]{shr}
\DeclareGraphicsExtensions{.pdf,.jpg,.png}

\item You can view your progress in the progress bar.
\newline
\includegraphics[totalheight=7cm]{progress}
\DeclareGraphicsExtensions{.pdf,.jpg,.png}
\end{enumerate}

\newpage 

\section{Steps to be followed by the teacher}

\begin{enumerate}
\item Create a course.Go o its advanced settings and add the following xblock 'agea' in its advanced module list.\\
\includegraphics[totalheight=7cm]{1}
\DeclareGraphicsExtensions{.pdf,.jpg,.png}

\item Now go back to your course and create sections,subsections and units as required.On selecting a particular unit,
you need to choose your required XBlock.In our case,it is 'Excel Autograded Assignment'.\\
\includegraphics[totalheight=3cm]{8}
\DeclareGraphicsExtensions{.pdf,.jpg,.png}

\newpage 

\item Now, the teacher is shown a default structure of the student view.Click on the edit button at the top right corener.\\
\includegraphics[totalheight=10cm]{start}
\DeclareGraphicsExtensions{.pdf,.jpg,.png}

\item Now,you get need to fill in all the input fields,ie the Question Title,Question Text,Maximum Score,
Problem Weight and Maximum atempts.\\
\includegraphics[totalheight=7cm]{teach3}
\DeclareGraphicsExtensions{.pdf,.jpg,.png}

\newpage 

\item Since you have not uploaded any question file yet,the following is displayed.\\
\includegraphics[totalheight=5cm]{t}
\DeclareGraphicsExtensions{.pdf,.jpg,.png}

\item Select a suitable question file using the browse button.Say,the name of your question file is annotated.xlsx.\\
\includegraphics[totalheight=5cm]{bla}
\DeclareGraphicsExtensions{.pdf,.jpg,.png}

\item Upload a suitable question file.You can also change it or download it if you want to.\\
\includegraphics[totalheight=6cm]{te}
\DeclareGraphicsExtensions{.pdf,.jpg,.png}

\newpage

\item In a similar manner upload your solution file.\\
\includegraphics[totalheight=6cm]{teacher2}
\DeclareGraphicsExtensions{.pdf,.jpg,.png}

\item Save and publish your changes.

\item You can view the grades of the students in the grade book.\\
\includegraphics[totalheight=6cm]{grades}
\DeclareGraphicsExtensions{.pdf,.jpg,.png}
\end{enumerate}

\section{User Requirement}
The Student and the Teacher are expected to be Internet literate and be able to use a
search engine.

\chapter{Limitations}
\begin{enumerate}
\item The grading is based on cell comparison only.

\item All the files uploaded should be in xlsx format.

\item Also,there will be error if we duplicate sections or subsections.

\item There should not be any videos or pictures in the excel file.

\item The author view is not working.That is why we have shown the instructions(originally 
intented to be the content of the author view) is shown in the studio view.

\end{enumerate}
\chapter{Appendix}

\section{Description of the various Fields}

\subsection{display_name}
\subsubsection{Purpose:}
It displays the name of the course at the top of the student view and studio view.
\subsubsection{Scope:}
Scope.settings

\subsection{question}
\subsubsection{Purpose:}
This is the problem statement given by the teacher.It is shown in student's view and it tells the student what to do.
\subsubsection{Scope:}
Scope.settings

\subsection{weight}
\subsubsection{Purpose:}
It defines the number of points each problem is worth.If the value is not set, the problem is worth the sum of the
        option point values.For example,if weight = 100, then the exam will be graded out of 100.
\subsubsection{Scope:}
Scope.settings

\subsection{points}
\subsubsection{Purpose:}
It is the maximum score that can be obtained by a student on a particular assignment.
\subsubsection{Scope:}
Scope.settings

\subsection{score}
\subsubsection{Purpose:}
It is the score given to the student.It is calculated by the autograder.
\subsubsection{Scope:}
Scope.user_state

\subsection{attempts}
\subsubsection{Purpose:}
No. of times the student uploads the answer file.
\subsubsection{Scope:}
Scope.user_state

\subsection{max_attempts}

\subsubsection{Purpose:}
No. of times the student is allowed to upload the answer file.If this field is not filled by the teacher,then the 
student can submit the answer file infinite nunber of times. 
\subsubsection{Scope:}
Scope.settings


\subsection{raw_answer}

\subsubsection{Purpose:}
It refers to the answer excel file uploaded by the student.
\subsubsection{Scope:}
Scope.user_state

\subsection{raw_question}
\subsubsection{Purpose:}
It refers to the question excel file uploaded by the teacher.
\subsubsection{Scope:}
Scope.settings

\subsection{raw_solution}
\subsubsection{Purpose:}
It refers to the solution excel file uploaded by the teacher.
\subsubsection{Scope:}
Scope.settings


\section{Description of the Class Methods}


\subsection{get_submission(self)}
It returns the class field raw_answer.

\subsection{get_question(self)}
It returns the class field raw_question.

\subsection{get_solution(self)}
It returns the class field raw_solution.

\subsection{student_view(self, context=None)}
It renders the content of cms.It stores student_state,id and max_file_size in context and passes it to the html file.It integrates
the html,js and css files in a variable fragment and returns it.

\subsection{studio_view(self, context=None)}

It stores display_name,question,points,weight and max_attempts in edit_fields.It stores edit_fields in context along with 
studio_state,max_file_size and id and passes it to the html file.It integrates
the html,js and css files in a variable fragment and returns it.

\subsection{studio_state(self)}
It returns the following to be used in studio_view::
1.display_name\\
2.self.question\\
3.quploaded\\
4.suploaded\\
5.self.raw_question\\
6.suploaded\\
7.self.raw_solution\\
8.self.weight

\subsection{student_state(self)}
It returns the following to be used in studio_view::
1.display_name\\
2.self.question\\
3.uploaded\\
4.raw_answer\\
5.raw_question\\
6.score\\
7.weight\\
8.attempts\\
9.max_attempts


\section{Description of the Miscellaneous Methods}


\subsection{_get_sha1(file_descriptor)}
It returns a hash value that is unique for every file.

\subsection{_resource(path)}
It is used to render the template. 

\subsection{load_resource(resource_path)}
It is used to load the template.

\subsection{render_template(template_path, context=None)}
It is used to render the template.

\end{document}